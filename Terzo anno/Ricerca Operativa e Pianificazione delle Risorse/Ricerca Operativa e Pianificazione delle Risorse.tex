\documentclass[11pt,a4paper]{book}
<<<<<<< HEAD
\usepackage{Appunti_universitari}
=======
\usepackage[italian]{babel}
\usepackage[T1]{fontenc}
\usepackage[utf8]{inputenc}
\usepackage{graphicx}
\usepackage{imakeidx}
\usepackage{amsmath}
\usepackage[hyperfootnotes=false, colorlinks=true, linkcolor=black]{hyperref}
\usepackage[style=numeric-comp,useprefix,hyperref,backend=bibtex]{biblatex}

>>>>>>> 522e83e... Aggiornato C++

\begin{document}
\title{Ricerca Operativa e Pianificazione delle Risorse}
\author{Jacopo De Angelis}
\maketitle

\pagebreak
\tableofcontents
\pagebreak

\begin{LARGE}
Programma esteso
\end{LARGE}

\begin{itemize}
\item Introduzione: storia-motivazione-esempi

    \item Ottimizzazione non lineare
    \begin{itemize}
        \item Ottimizzazione di funzioni non lineari ad una variabile: ricerca dicotomia-metodo Bisezione- metodo Newton
        \item Ottimizzazione di funzioni non lineari mutivariate: metodo Gradiente-metodo Newton
        \item Ottimizzazione non lineare vincolata: condizioni di Karush-Kuhn-Tucker
    \end{itemize}

    \item Ottimizzazione lineare
    \begin{itemize}
        \item Introduzione alla programmazione lineare (PL): proprietà dei problemi di PL, strategie di modellizzazione
        \item Soluzione grafica: soluzione grafica per problemi di PL
        \item Geometria della Programmazione lineare e metodo del simplesso
        \item Dualità e analisi di sensitività
        \item Problemi di PL con variabili binarie e problemi di PL Intera e mista: formulazione problemi e metodo del Branch & Bound
    \end{itemize}

    \item Soft Computing per l'ottimizzazione
    \begin{itemize}
        \item Algoritmi evolutivi
        \item Reti neurali ed SVM
    \end{itemize}
\end{itemize}
\pagebreak

<<<<<<< HEAD
\end{document}
=======
>>>>>>> 522e83e... Aggiornato C++
