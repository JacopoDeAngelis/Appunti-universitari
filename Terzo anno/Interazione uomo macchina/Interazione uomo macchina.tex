\documentclass[11pt,a4paper]{book}
\usepackage{Appunti_universitari}

\begin{document}
\title{Interazione uomo macchina}
\author{Jacopo De Angelis}
\maketitle

\pagebreak
\tableofcontents
\pagebreak

\chapter{Modulo 1: Concetti di base}
\section{Cos'è l'interazione uomo macchina?}
\begin{center}
Definizione standard 

\textit{"HCI (Human-Computer Interaction) è una disciplina che si occupa della progettazione, realizzazione e valutazione di sistemi interattivi con capacità computazionali destinati all'uso umano e dello studio dei principali fenomeni che li circondano"} - \textit{Associacion for Computing Machinery}
\end{center}

L'usabilità di un sistema è spesso trascurata all'interno dell'ambito lavorativo italiano, non per faciloneria ma perchè le risorse da adibire a questo ambito sono una spesa che non viene vista come essenziale nella produzione del valore per andare avanti. 

Il problema viene aggravato dall'outsourcing verso paesi dove il costo del lavoro sia inferiore e, ultimamente, anche da sistemi di ML che sono in grado, con grado di precisione sempre maggiore grazie al deep learning, di sviluppare codice funzionante. Ma cosa viene assegnato ai lavoratori esterni? Ciò che è altamente formalizzato, in modo da lasciare poche possibilità di variazione dalle richieste dei clienti.

Cosa rimane difficile da esternalizzare? Il contatto del cliente, sia durante la prima raccolta di requisiti funzionali, sia le successive interazioni con esso per cambiamenti incrementativi, variazioni di funzionalità o feedback.

L'usabilità è quella caratteristica che rende "facile la vita" al cliente.

\noindent\rule{\textwidth}{1pt}
\begin{center}
\textit{Piccola digressione}
\end{center}
\textbf{Perchè la concorrenza moderna rende sempre più importante l'analisi dell'interazione uomo macchina?}

In un contesto monopolistico le aziende non sono invogliate a produrre la soluzione "migliore"\footnote{Dove per migliore si intende quella che prende in considerazione più metriche come usabilità, efficienza, efficacia, design, ecc} ma solo quella più efficiente a livello di costo marginale.   

Cosa vuole dire questo? Che nella moderna concorrenza derivante da sistemi di sviluppo sempre più semplici e un'offerta più rapida tramite internet, per ottenere quote di mercato le aziende devono iniziare a pensare non solo al "funziona?" ma anche al "come lo faccio funzionare?"
\\\noindent\rule{\textwidth}{1pt}

\textbf{"Qui non si impara a fare delle interfacce usabili, si impara a riconoscere l'usabilità delle interfacce"} ovvero impariamo strumenti che ci possono portare a fare delle belle interfacce, certo, ma soprattutto ci permette di riconoscere cosa renda \underline{buona} un'interfaccia.

La disciplina nasce negli anni '80 ma l'interazione con le macchine (intese come calcolatori) esiste dagli anni'40, semplicemente prima l'utente era ultra specializzato mentre ora quasi tutti possono accedere ad un PC e con questo interagire tramite un'interfaccia.

Dal 1983 si tiene la conferenza annuale \href{https://dl.acm.org/conference/chi}{ACM CHI}. In questi casi tre aree disciplinari si incontrano:
\begin{itemize}
	\item ergonomia
	\item informatica
	\item psicologia comportamentista
\end{itemize}
\begin{figure}[h!]
	\begin{center}
		\includegraphics[scale=0.6]{img/001.jpg}
		\caption{HCI e le sue componenti}
		\label{fig: 001}
	\end{center}
\end{figure}
Dobbiamo ricordare sempre una cosa: dobbiamo lasciarci alle spalle l'utente ideale, l'utente senza faccia e senza capacità, e iniziare a pensare all'utente reale, ovvero a chi, idealmente, è diretta la nostra interfaccia. \underline{L'utente non siamo noi}. Proprio per questo le interviste, i test con esterni, i mockup sono utili, perchè ci permettono di vedere la nostra idea attraverso gli occhi di altri. Esempio banale: la nostra interfaccia basata sul colore potrebbe essere altamente confusionaria per un daltonico. Un'applicazione mobile dove tutti i comandi sono sulla destra potrebbe essere difficile da usare per un mancino.

\textbf{Ergonomia cognitiva}: studio dell'interazione tra l'uomo e gli strumenti per l'elaborazione di informazioni studiando i processi cognitivi coinvolti (percezione, attenzione, memoria, pensiero, linguaggio, emozioni) e suggerendo delle soluzioni per migliorare tali strumenti.

\section{Perchè è difficile progettare perchè l'interazione sia "buona"?}
Ci sono tre ragioni, idealmente:
\begin{enumerate}
	\item \textbf{La varietà dei sistemi interattivi}: cellulari, computer, cloche, macchine da cucina, tablet...
	\item \textbf{La varietà degli utenti}: fasce d'età, background culturale, condizioni mediche...
	\item \textbf{La varietà degli scopi e degli usi}: contesti formativi, ludici, lavorativi e usi tramite dispositivi diversi, in luoghi differenti...
\end{enumerate}

\begin{center}
\textbf{\textit{Noi studieremo come progettare per la varietà e al volto delle procedure.}}
\end{center}

\end{document}