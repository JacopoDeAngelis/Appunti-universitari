\documentclass[11pt,a4paper]{book}
\usepackage[italian]{babel}
\usepackage[T1]{fontenc}
\usepackage[utf8]{inputenc}
\usepackage{graphicx}
\usepackage{imakeidx}
\usepackage{amsmath}
\usepackage[hyperfootnotes=false, colorlinks=true, linkcolor=black]{hyperref}
\usepackage[style=numeric-comp,useprefix,hyperref,backend=bibtex]{biblatex}


\begin{document}
\title{Programmazione C++}
\author{Jacopo De Angelis}
\maketitle

\pagebreak
\tableofcontents
\pagebreak

\begin{LARGE}
Programma esteso
\end{LARGE}

\begin{itemize}
	\item Introduzione al C++.
	\item Concetti base di programmazione C++
	\begin{itemize}
		\item tipi di dati, puntatori,  reference, scoping
		\item casting,
	\end{itemize}

	\item C++ come linguaggio ad oggetti
	\begin{itemize}
		\item classi, costruttori e distruttori, overloading, metodi friend
		\item inline, constness"           
	\end{itemize}	
	\item Concetti avanzati di programmazione C++
	\begin{itemize}
		\item overloading degli operatori
		\item metodi virtual, abstract, polimorfismo
		\item ereditarietà
	\end{itemize}
	\item Programmazione generica
	\begin{itemize}
		\item template
		\item iteratori
	\end{itemize}	
	\item La libreria Standard (STL)
	\begin{itemize}
		\item Le classi container
		\item Gli algoritmi
		\item Funtori
		\item Multithread
	\end{itemize}	
	\item Uso delle librerie esterne
	\begin{itemize}
		\item Librerie statiche
		\item Librerie dinamiche
		\item La libreria OpenMP
	\end{itemize}	
	\item I nuovi standard C++11, C++14
	
	\item Applicazioni GUI
	\begin{itemize}
		\item Ambiente di sviluppo QT Creator
		\item Sviluppo di interfacce grafiche
		\item Gestione degli eventi
		\item Le librerie Qt, QTWidgetscontenuto...
	\end{itemize}
\end{itemize}
\pagebreak

\chapter{Introduzione}
\section{Il design di C++}
C++ è basato su due idee:
\begin{itemize}
    \item mapping diretto delle operazoini built-in e dei tipi di hardware in modo da fornire un uso efficiente della memoria e operazioni a basso livello;
    \item astrazioni flessibili e accessibili per fornire dei tipi definiti dall'utente
\end{itemize}

