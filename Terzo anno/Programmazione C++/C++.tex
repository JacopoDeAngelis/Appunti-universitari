\documentclass[11pt,a4paper]{book}
\usepackage[margin=2in]{geometry}
\usepackage[italian]{babel}
\usepackage[T1]{fontenc}
\usepackage[utf8]{inputenc}
\usepackage{graphicx}
\usepackage{imakeidx}
\usepackage{amsmath}
\usepackage{color}


\usepackage[hyperfootnotes=false, colorlinks=true, linkcolor=black]{hyperref}
\usepackage[style=numeric-comp,useprefix,hyperref,backend=bibtex]{biblatex}
\usepackage{listings}  % Serve per evidenziare i blocchi di codice
\usepackage{pxfonts} % permette di avere caratteri in lstlisting con formattazione

\setlength{\parskip}{1em} % cambia l'interlinea prima di un nuovo capoverso

% Colori per lstlisting
\definecolor{pblue}{rgb}{0.13,0.13,1}
\definecolor{pgreen}{rgb}{0,0.5,0}
\definecolor{pred}{rgb}{0.9,0,0}
\definecolor{pgrey}{rgb}{0.46,0.45,0.48}
\definecolor{maroon}{rgb}{0.5,0,0}
\definecolor{plightgrey}{rgb}{0.8,0.8,0.8} 
\definecolor{darkblue}{rgb}{0.0,0.0,0.6}
\definecolor{cyan}{rgb}{0.0,0.6,0.6}

\usepackage{xcolor} % Necessario per definire i colori
\hypersetup{
  colorlinks=true,
  linkcolor=red!70!black,
  urlcolor=blue!70!black
} % Setup colore link


\lstset{ % Riduce la larghezza della tabulazione per lstlisting
  tabsize=2,
  backgroundcolor=\color{plightgrey},
  breaklines=true,
  postbreak=\mbox{\textcolor{red}{$\hookrightarrow$}\space},
  columns=fullflexible,
  frame=single,
}


\lstset{
  language=XML,
  basicstyle=\ttfamily,
  morestring=[s]{"}{"},
  morecomment=[s]{?}{?},
  morecomment=[s]{!--}{--},
  commentstyle=\color{white},
  moredelim=[s][\color{black}]{>}{<},
  moredelim=[s][\color{red}]{\ }{=},
  stringstyle=\color{blue},
  identifierstyle=\color{maroon}
}
% Crea un set di impostazioni per Java in lstlisting
\lstset{language=Java,
  showspaces=false,
  showtabs=false,
  breaklines=true,
  showstringspaces=false,
  breakatwhitespace=true,
  commentstyle=\color{white},
  keywordstyle=\color{pblue},
  stringstyle=\color{pred},
  basicstyle=\ttfamily,
  moredelim=[is][\textcolor{pgrey}]{\%\%}{\%\%}
}

% Crea un set di impostazioni per C++ in lstlisting
\lstset{language=C++,
                basicstyle=\ttfamily,
                keywordstyle=\color{blue}\ttfamily,
                stringstyle=\color{red}\ttfamily,
                commentstyle=\color{white}\ttfamily,
                morecomment=[l][\color{magenta}]{\#}
}

\lstdefinelanguage{XML}{
  morestring=[b]",
  morestring=[s]{>}{<},
  morecomment=[s]{<?}{?>},
  stringstyle=\color{black},
  identifierstyle=\color{darkblue},
  keywordstyle=\color{cyan},
  morekeywords={xmlns,version,type}% list your attributes here
}

\begin{document}
\title{Programmazione C++ \\
	\large Riassunto di "The C++ Programming Language" \\
		4a edizione \\
	 	\textit{Bjarne Stroustrup}
}
\author{Jacopo De Angelis}
\maketitle

\pagebreak
\tableofcontents
\pagebreak

\begin{LARGE}
Programma esteso
\end{LARGE}

\begin{itemize}
	\item Introduzione al C++.
	\item Concetti base di programmazione C++
	\begin{itemize}
		\item tipi di dati, puntatori,  reference, scoping
		\item casting,
	\end{itemize}

	\item C++ come linguaggio ad oggetti
	\begin{itemize}
		\item classi, costruttori e distruttori, overloading, metodi friend
		\item inline, constness"           
	\end{itemize}	
	\item Concetti avanzati di programmazione C++
	\begin{itemize}
		\item overloading degli operatori
		\item metodi virtual, abstract, polimorfismo
		\item ereditarietà
	\end{itemize}
	\item Programmazione generica
	\begin{itemize}
		\item template
		\item iteratori
	\end{itemize}	
	\item La libreria Standard (STL)
	\begin{itemize}
		\item Le classi container
		\item Gli algoritmi
		\item Funtori
		\item Multithread
	\end{itemize}	
	\item Uso delle librerie esterne
	\begin{itemize}
		\item Librerie statiche
		\item Librerie dinamiche
		\item La libreria OpenMP
	\end{itemize}	
	\item I nuovi standard C++11, C++14
	
	\item Applicazioni GUI
	\begin{itemize}
		\item Ambiente di sviluppo QT Creator
		\item Sviluppo di interfacce grafiche
		\item Gestione degli eventi
		\item Le librerie Qt, QTWidgetscontenuto...
	\end{itemize}
\end{itemize}
\pagebreak

\chapter{Principi base}
\section{Le basi}





\end{document}