\documentclass[11pt,a4paper]{book}
\usepackage{Appunti_universitari}

\begin{document}
\title{Programmazione dispositivi mobili}
\author{Jacopo De Angelis}
\maketitle

\pagebreak
\tableofcontents
\pagebreak

\begin{LARGE}
Programma esteso
\end{LARGE}

\begin{itemize}
	\item \textbf{Introduzione alla progettazione e allo sviluppo di applicazoni mobili}
	\begin{itemize}
		\item Sviluppo di applicazioni mobili
		\item Opportunità di mercato
		\item Requisiti tecnici per Apple (iOS), Google (Android)
		\item Sfide: dimensioni limitate dello schermo, problemi di memoria e frammentazione
		\item Cenni a framework cross-platform (e.g., Flutter, React Native, Apache Cordoba, Sencha, Corona, Xamarin)
	\end{itemize}
	\item \textbf{Progettazione dell'interfaccia utente}
	\begin{itemize}
		\item Linee guida per la progettazione di un'interfaccia utente non solo bella, ma anche usabile (Material Design)
		\item Le persone al primo posto: parametri da considerare per rendere accessibile a tutti la propria applicazione
		\item Progettazione Mobile First e Responsive Design
		\item Imparare ad usare i colori, i font, e in generale i componenti grafici più appropriati per ogni contesto
	\end{itemize}
	\item \textbf{Sviluppo di applicazioni per dispositivi Android}
	\begin{itemize}
		\item Introduzione alla piattaforma Android
		\item Ambiente di sviluppo: Android Studio, Google Software Development kit e le versioni, Genymotion e debugger
		\item Ciclo di vita di un'applicazione: le Activity e i Fragment
		\item Layout e widget di base ed avanzati
		\item La concorrenza: threads e task asincroni
		\item Oltre l'aspetto grafico: Content provider e Service
		\item Architettura di un'applicazione Android: Model-View-Presenter
		\item Pubblicazione di un'applicazione sul Google Play Store
	\end{itemize}
	\item  \textbf{Sviluppo di Web Application}
	\begin{itemize}
		\item Introduzione al concetto di Web App: backend e frontend
		\item Ambiente di sviluppo: editor, strumenti per sviluppatori (Chrome e Firefox), documentazione online (MDN)
		\item Introduzione all'HTML: definire la struttura di una pagina Web
		\item Introduzione al CSS: definire lo stile di una pagina Web
		\item Introduzione a JavaScript: definire la logica associata ad una pagina Web
		\item Utilizzo di un framework per la realizzazione di una Web App (e.g, Bootstrap)
		\item Deployment di una Web App	
	\end{itemize}
\end{itemize}


\end{document}