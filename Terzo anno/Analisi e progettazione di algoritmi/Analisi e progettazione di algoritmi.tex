\documentclass[11pt,a4paper]{book}
<<<<<<< HEAD
\usepackage{Appunti_universitari}
=======
\usepackage[italian]{babel}
\usepackage[T1]{fontenc}
\usepackage[utf8]{inputenc}
\usepackage{graphicx}
\usepackage{imakeidx}
\usepackage{amsmath}
\usepackage[hyperfootnotes=false, colorlinks=true, linkcolor=black]{hyperref}
\usepackage[style=numeric-comp,useprefix,hyperref,backend=bibtex]{biblatex}

>>>>>>> 522e83e... Aggiornato C++

\begin{document}
\title{Analisi e progettazione di algoritmi}
\author{Jacopo De Angelis}
\maketitle

\pagebreak
\tableofcontents
\pagebreak

\begin{LARGE}
Programma esteso
\end{LARGE}

\begin{itemize}
    \item Strumenti matematici
    \begin{itemize}
        \item Crescita delle funzioni, notazioni asintotiche
        \item Calcolo del tempo di esecuzione per algoritmi iterativi
        \item Richiami sulla ricorsione: calcolo del fattoriale
        \item Ricorrenze e tempi di calcolo di algoritmi ricorsivi
        \item Ricerca dicotomica, calcolo altezza di un albero binario
    \end{itemize}

    \item Tecniche algoritmiche: Programmazione Dinamica (DP)
    \begin{itemize}
        \item Esempi introduttivi
        \item Caratteristiche principali - Ricorsione
        \item mplementazione con matrici   
    \end{itemize}

    \item Tecniche algoritmiche: il metodo Greedy (goloso)
    \begin{itemize}
        \item Esempi introduttivi
        \item I codici di Huffman
        \item Matroidi
        \item Teorema di Rado  
    \end{itemize}

    \item Algoritmi su grafi
    \begin{itemize}
        \item Rappresentazione dei grafi.
        \item Visita in ampiezza dei grafi
        \item Visita in profondità dei grafi    
    \end{itemize}

    \item Alberi di copertura minimi
    \begin{itemize}
        \item Algoritmo di Kruskal
        \item Algoritmo di Prim   
    \end{itemize}

    \item Problemi di cammino minimo
    \begin{itemize}
        \item Algoritmo di Dijkstra
        \item Algoritmo di Bellman-Ford
        \item Algoritmo di Floyd-Warshall
    \end{itemize

    \item Problemi di flusso massimo
    \begin{itemize}
        \item Algoritmo di Ford-Fulkerson
    \end{itemize}

    \item NP completezza e riducibilità
    \begin{itemize}
        \item P versus NP
        \item dimostrazioni di NP completezza
        \item Problemi NP completi
    \end{itemize}
\end{itemize}
\pagebreak


<<<<<<< HEAD
\end{document}
=======
>>>>>>> 522e83e... Aggiornato C++
